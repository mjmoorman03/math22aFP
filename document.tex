\documentclass{article}
\usepackage[utf8]{inputenc}
\usepackage[english]{babel}
\usepackage{amsmath}
\usepackage{graphicx}
\graphicspath{{./images}}
\usepackage{amsthm}
\usepackage{marvosym}
\let\marvosymLightning\Lightning
\usepackage{amssymb}
\usepackage[dvipsnames]{xcolor}
\usepackage[framemethod=TikZ]{mdframed}
\newcommand{\R}{\mathbb{R}}
\newcommand{\Z}{\mathbb{Z}}
\newcommand{\N}{\mathbb{N}}
\newcommand{\done}{\renewcommand\qedsymbol{$\blacksquare$}}
\newcommand{\contradiction}{\renewcommand\qedsymbol{$\Lightning$}}
\usepackage[left=3cm,right=3cm,top=2cm,bottom=2cm]{geometry} % page settings
\newtheorem{lemma}{Lemma}
\newtheorem{sublemma}{Lemma}[section]
% augmented matrix environment
\newenvironment{amatrix}[1]{%
  \left[\begin{array}{@{}*{#1}{c}|c@{}}
}{%
  \end{array}\right]
}

\title{\textbf{A Swift Introduction to Group Theory}}
\author{Cade McManus, Michael Moorman}
\date{December 2022}

\begin{document}

\maketitle

\section{What is Group Theory?}

In order to connect the ideas of group theory to the real world, we must first understand what a group is 
in relation to linear algebra. A group is a set of elements that are closed under a binary operation,
similarly to how vector spaces are closed under addition and scalar multiplication, but with the 
generalization that groups need not be composed of only vectors, but any set of elements.

\paragraph*{Definition 1:} A \textbf{group} is a set $S$ with a binary\footnote[1]{\textbf{Binary} refers to the number of elements that the operation requires to be defined, specifically 2. Consider the binary operation of multiplication. The expression $5*$ does not make sense without a second input, making multiplication of real numbers a binary operation. }
operation $\cdot$ 
such that the following axioms hold:
% axioms of a group
\begin{enumerate}
    \item Closure: For all $a,b \in S$, $a \cdot b \in S$.
    \item Associativity: For all $a,b,c \in S$, $(a \cdot b) \cdot c = a \cdot (b \cdot c)$.
    \item Identity: There exists a unique element $e \in S$ such that for all $a \in S$, $a \cdot e = e \cdot a = a$.
    \item Inverse: For all $a \in S$, there exists a unique element $a^{-1} \in S$ such that $a \cdot a^{-1} = a^{-1} \cdot a = e$.
\end{enumerate}
Formally, the set $S$ is known as the \textbf{underlying set} of the group, and 
the binary operation $\cdot$ is known as the \textbf{group operation}. Frequently,
a group $G$ is denoted by $G = \langle S, \cdot \rangle$. Additionally, abuses 
of notation are common, such that $G$ referring to a group can be used to refer 
to the group itself or the underlying set, such that a statement like '$a\in G$'
is meant to convey that $a$ is an element in the underlying set of the group $G$.

Additionally, another special type of group we will see is an \textbf{abelian group}.

\textbf{Definition 2:} An \textbf{abelian group} is a group $\langle V, \cdot \rangle$
such that for all $u,v \in V$, $u\cdot v = v\cdot u$. This means that our group operation 
is commutative, and the order in which we add elements does not matter.

\paragraph*{Example:}
This may seem confusing, but let's make it more concrete with an example: 
the real numbers together with the operation of addition as we know it. This 
group would be written $G = \langle \R, +\rangle$. Let's verify our axioms 
to truly grasp what they mean through this example:
\begin{enumerate}
    \item Closure: Let $x,y\in\R$. Because the sum of any two real numbers 
    is defined as a real number, it must be the case that $x+y\in\R$. 
    \item Associativity: Let $x,y,z\in\R$. Because addition of real numbers 
    does not depend on the way in which terms are associated, it is the case 
    that $(x+y)+z = x+y+z$ and that $x+(y+z)=x+y+z$. Thus, $(x+y)+z=x+(y+z)$. 
    \item Identity: Let $e = 0$. $0\in\R$, and the addition of 0 to any $x\in\R$
    is equal to $x$ itself. Thus, for all $x\in\R$, $e+x=x+e=x$. 
    \item Inverse: For all $x\in\R$, allow $x^{-1}$, the inverse of $x$, to be equal 
    to $-x$, the negation of $x$. Thus, $-x+x=0$ via simple addition, and $e=0$.
    Therefore, for any choice of $x\in\R$, there exists an inverse element $x^{-1}$ in $\R$
    such that $x+ x^{-1} =e$, the identity element. 
\end{enumerate}
That's all! We have proven that the real numbers, along with the operation of 
addition as we know it, form a group! Now that we're a bit more comfortable with 
what a group is, what does it have to do with linear algebra? Or even anything 
else at all?

\section{Vector Spaces and their Connections to Groups}

In class, we defined a vector space as a non-empty set $V$ on which two operations,
scalar multiplication and vector addition, are defined subject to the following axioms,
where $u,v,w \in V$ and $c,d \in \R$:
\begin{enumerate}
    \item $u+v \in V$
    \item $u+v = v+u$
    \item $(u+v)+w = u+(v+w)$
    \item there exists a vector $0_v \in V$ such that $u+0_v = u$
    \item there exists a vector $-u \in V$ such that $u+(-u) = 0_v$
    \item $cu\in V$
    \item $c(u+v) = cu+cv$
    \item $(c+d)u = cu+du$
    \item $c(du) = (cd)u$
    \item $1 \cdot u = u$
\end{enumerate}
This is absolutely a correct definition, but we can introduce some additional 
notation as well as generalize $\R$. We will say that a vector space is a tuple 
$\langle V, K, +, \cdot \rangle$, where $V$ is a non-empty set, $K$ is a field\footnote[2]{A \textbf{field} refers to a set on which addition, multiplication, subtraction, etc. are defined as we know them to work for the real numbers. This is not incredibly important here, but it is relevant to note that the field $K$ is not necessarily the real numbers.},
$+$ is our vector addition operator, and $\cdot$ is our scalar multiplication operator,
and all of the previous axioms hold.

As we have seen, a group is a set of elements closed under a binary operation, which 
seems like a less restrictive version of a vector space, especially after having 
established this notation. It turns out that this is indeed the case, and that 
given our vector field $\langle V, K, +, \cdot \rangle$, we can always define 
an abelian group as simply $\langle V, + \rangle$.

\begin{mdframed}[roundcorner=10pt, backgroundcolor=gray!10]
  \textbf{Theorem 1:} Let $\langle V, K, +, \cdot \rangle$ be a vector space. Then,
  $\langle V, + \rangle$ is an abelian group.
\end{mdframed}

\begin{proof}
    In order to prove this theorem, we will verify all of the axioms of a group
    given that $V$ is a non-empty set of a vector space, and $+$ is the vector 
    addition operator from that same vector space. 

\done  
\end{proof}


\section{Group Homomorphisms} 


\section{Subgroups}

\end{document}