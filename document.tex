\documentclass{article}
\usepackage[utf8]{inputenc}
\usepackage[english]{babel}
\usepackage{amsmath}
\usepackage{graphicx}
\graphicspath{{./images}}
\usepackage{amsthm}
\usepackage{marvosym}
\let\marvosymLightning\Lightning
\usepackage{amssymb}
\newcommand{\R}{\mathbb{R}}
\newcommand{\Z}{\mathbb{Z}}
\newcommand{\N}{\mathbb{N}}
\newcommand{\done}{\renewcommand\qedsymbol{$\blacksquare$}}
\newcommand{\contradiction}{\renewcommand\qedsymbol{$\Lightning$}}
\usepackage[left=3cm,right=3cm,top=2cm,bottom=2cm]{geometry} % page settings
\newtheorem{lemma}{Lemma}
\newtheorem{sublemma}{Lemma}[section]
% augmented matrix environment
\newenvironment{amatrix}[1]{%
  \left[\begin{array}{@{}*{#1}{c}|c@{}}
}{%
  \end{array}\right]
}

\title{\textbf{A Swift Introduction to Group Theory}}
\author{Cade McManus, Michael Moorman}
\date{December 2022}

\begin{document}

\maketitle

\section{What is Group Theory?}

In order to connect the ideas of group theory to the real world, we must first understand what a group is 
in relation to linear algebra. A group is a set of elements that are closed under a binary operation,
similarly to how vector spaces are closed under addition and scalar multiplication, but with the 
generalization that groups need not be composed of only vectors, but any set of elements.

\paragraph*{Definition 1:} A \textbf{group} is a set $G$ with a binary operation $\cdot$ 
such that the following axioms hold:
% axioms of a group
\begin{enumerate}
    \item Closure: For all $a,b \in G$, $a \cdot b \in G$.
    \item Associativity: For all $a,b,c \in G$, $(a \cdot b) \cdot c = a \cdot (b \cdot c)$.
    \item Identity: There exists an element $e \in G$ such that for all $a \in G$, $a \cdot e = e \cdot a = a$.
    \item Inverse: For all $a \in G$, there exists an element $a^{-1} \in G$ such that $a \cdot a^{-1} = a^{-1} \cdot a = e$.
\end{enumerate}

\end{document}